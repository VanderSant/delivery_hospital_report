\documentclass[]{politex}
% ========== Opções ==========
% pnumromarab - Numeração de páginas usando algarismos romanos na parte pré-textual e arábicos na parte textual
% abnttoc - Forçar paginação no sumário conforme ABNT (inclui "p." na frente das páginas)
% normalnum - Numeração contínua de figuras e tabelas 
%	(caso contrário, a numeração é reiniciada a cada capítulo)
% draftprint - Ajusta as margens para impressão de rascunhos
%	(reduz a margem interna)
% twosideprint - Ajusta as margens para impressão frente e verso
% capsec - Forçar letras maiúsculas no título das seções
% espacosimples - Documento usando espaçamento simples
% espacoduplo - Documento usando espaçamento duplo
%	(o padrão é usar espaçamento 1.5)
% times - Tenta usar a fonte Times New Roman para o corpo do texto
% noindentfirst - Não indenta o primeiro parágrafo dos capítulos/seções


% ========== Packages ==========
\usepackage[utf8]{inputenc}
\usepackage{amsmath,amsthm,amsfonts,amssymb}
\usepackage{graphicx,cite,enumerate}


% ========== Language options ==========
\usepackage[brazil]{babel}
%\usepackage[english]{babel}


% ========== ABNT (requer ABNTeX 2) ==========
%	http://www.ctan.org/tex-archive/macros/latex/contrib/abntex2
%\usepackage[num]{abntex2cite}

% Forçar o abntex2 a usar [ ] nas referências ao invés de ( )
%\citebrackets{[}{]}


% ========== Lorem ipsum ==========
\usepackage{blindtext}



% ========== Opções do documento ==========
% Título
\titulo{Robô Hospitalar}

% Autor
\autor{Vanderson da Silva dos Santos}

% Para múltiplos autores (TCC)
%\autor{Nome Sobrenome\\%
%		Nome Sobrenome\\%
%		Nome Sobrenome}

% Orientador / Coorientador
\orientador{Leopoldo Rideki Yoshioka}
%\coorientador{Nome do coorientador (opcional)}

% Tipo de documento
\tcc{}
%\dissertacao{Engenharia Elétrica}
%\teseDOC{Engenharia Elétrica}
%\teseLD
%\memorialLD

% Departamento e área de concentração
\departamento{Engenharia de Sistemas Eletrônicos}
\areaConcentracao{Engenharia de Sistemas Eletrônicos}

% Local
\local{São Paulo}

% Ano
\data{2021}




\begin{document}
% ========== Capa e folhas de rosto ==========
\capa
\falsafolhaderosto
\folhaderosto


% ========== Folha de assinaturas (opcional) ==========
%\begin{folhadeaprovacao}
%	\assinatura{Prof.\ X}
%	\assinatura{Prof.\ Y}
%	\assinatura{Prof.\ Z}
%\end{folhadeaprovacao}


% ========== Ficha catalográfica ==========
% Fazer solicitação no site:
%	http://www.poli.usp.br/en/bibliotecas/servicos/catalogacao-na-publicacao.html


% ========== Dedicatória (opcional) ==========
\dedicatoria{Dedico esse trabalho aos amigos da Thunderatz e do Robô Hospitalar}


% ========== Agradecimentos ==========
\begin{agradecimentos}

Thanks...

\end{agradecimentos}


% ========== Epígrafe (opcional) ==========
\epigrafe{%
	\emph{``Mas nunca imaginaria que a curiosidade fosse outra dessas tantas ciladas do amor''}
	\begin{flushright}
		-{}- Gabriel Garcia Marquez
	\end{flushright}
}


% ========== Resumo ==========
\begin{resumo}
Resumo...
%
\\[3\baselineskip]
%
\textbf{Palavras-Chave} -- Palavra, Palavra, Palavra, Palavra, Palavra.
\end{resumo}


% ========== Abstract ==========
\begin{abstract}
Abstract...
%
\\[3\baselineskip]
%
\textbf{Keywords} -- Word, Word, Word, Word, Word.
\end{abstract}


% ========== Listas (opcional) ==========
\listadefiguras
\listadetabelas

% ========== Sumário ==========
\sumario

% ========== Elementos textuais ==========

% ========== Parte 1: Introdução ==========
\part{Introdução}
\chapter{Primeiro Robô Hospitalar}
\section{Apresentação}
\section{Problemas}
\chapter{Segundo Robô Hospitalar}
\section{Apresentação}
\section{Sistema Elétrico e Computacional}

% ========== Parte 2: Hardware ==========
\part{Hardware}
\chapter{Sistema Embarcado Completo}
\section{Apresentação}
\section{Alimentação}
\section{Comunicação}

\chapter{Placas de Circuito Impresso}
\section{Controle}
\subsection{Protótipo}
\subsection{Oficial}
\section{Percepção Externa}
\subsection{Protótipo}
\subsection{Oficial}
\section{Iluminação}
\subsection{Protótipo}
\subsection{Oficial}
\section{Telemetria}
\subsection{Protótipo}
\subsection{Oficial}
\section{Interface com Usuário}
\subsection{Protótipo}
\subsection{Oficial}

% ========== Parte 3: Software ==========
\part{Software}
\chapter{Sistema Computacional Completo}
\section{Apresentação}

\chapter{Ambiente de Simulação}
\section{Gazebo}
\section{Modelo do Robô}
\subsection{Carenagem}
\subsection{Sensores}
\section{Modelo do mundo}
\chapter{Algoritmos de Controle}
\section{ROS}
\subsection{Integração com Gazebo}
\section{Comunicação com Embarcados}

\blindtext

\begin{citacaoLonga}
	\blindtext
\end{citacaoLonga}

\blindtext


% ========== TITULOS DO SUMÁRIOS ==========
%\blinddocument
% =========================================


% ========== Referências ==========
% --- IEEE ---
%	http://www.ctan.org/tex-archive/macros/latex/contrib/IEEEtran
%\bibliographystyle{IEEEbib}

% --- ABNT (requer ABNTeX 2) ---
%	http://www.ctan.org/tex-archive/macros/latex/contrib/abntex2
%\bibliographystyle{abntex2-num}

%\bibliography{}


% ========== Apêndices (opcional) ==========
\apendice
\chapter{}
\chapter{Beta}


% ========== Anexos (opcional) ==========
\anexo
\chapter{Alpha}
\chapter{}



\end{document}
