\documentclass[../delivery_hospital_report.tex]{subfiles}
\graphicspath{ {images/}{../images/} } 

\begin{document}
\chapter{Arquitetura do Projeto}

O projeto da  robô hospitalar, desde a seus primórdios, tem uma arquitetura de projeto dividida em grandes três áreas de atuação, Sistemas Mecânicos, Eletrônicos e Computacionais. Para essa segunda versão do robô hospitalar, a sua estrutura de organização foi mantida, dessa forma, cada uma das áreas têm seus respectivos objetivos, funções e deveres para o projeto como um todo. Por mais que tenha essa divisão, os membros da equipe, que cursam entre engenharia elétrica, mecânica e mecatrônico, sempre acabam atuando em mais de uma área, o que contribui para a dinamicidade do projeto.

A área \textbf{Sistemas Mecânicos} é responsável pelo desenvolvimento, fabricação de modelos físicos, usinagem das peças, além da manutenção do robô como um todo. 

A área \textbf{Sistemas Eletrônicos} é destinada para a elaboração de todos os projetos da eletrônica embarcada, firmware, garantindo a alimentação, elétrica e segura, do robô e funcionamento dos componentes elétricos, em conjunto da transmissão de informação entre tais componentes e os sistemas computacionais.

A área \textbf{Sistemas Computacionais} tem como objetivo desenvolver algoritmos de controle, tomadas de decisão, construção de simuladores robóticas, visão computacional, desenvolvimento web e mobile.

Além dessas três áreas de atuação, há uma área não oficial de \textbf{design}, focada em divulgar a equipe e realizar a manutenção das redes sociais. \cite{site_robo_hospitalar21}

\begin{figure}[h]
\centering
    \caption{Sistema completo - Robô Hospitalar (V2)}
    \centering % para centralizarmos a figura
    \includegraphics[width=17cm]{sistema_robo.png}
    \caption*{Fonte: Elaborada pelo autor}
    \label{figura:1° Versão Robô Hospitalar}
\end{figure}

\section{Hardware}
Os sistemas eletrônicos do projeto tem como missão, antes de tudo, garantir a locomoção, iluminação, sensoriamento, telemetria modularizada e alimentação elétrica do robô hospitalar, sempre visando a precisão e a segurança. De maneira geral, os sistemas eletrônicos são divididos em duas frentes: Distribuição de energia e Módulos eletrônicos.

A frente de \textbf{distribuição de energia}, até Agosto de 2021, foi pouco explorada. Como um todo no projeto, há somente uma bateria com um tensão contínua elevada para alimentação total do robô e um regulador de tensão chaveado, que é necessário para diminuir a alta tensão da bateria e poder alimentar corretamente os módulos eletrônicos.

Por outro lado, a área dos \textbf{módulos eletrônicos} foi bem trabalhada no último ano. De maneira geral, é dividida em cinco grandes modelos eletrônicos: Sinalização, Percepção Externa, Telemetria, Controle e Interface com Usuário. Cada um desses módulos tem um parte totalmente eletrônica, que é composta por uma placa de circuito impresso, e uma parte do firmware do ESP32 \cite{esp32_datasheet}, o microcontrolador utilizado em cada uma das placas.

Por mais que cada placa tenha seu código único, muitas funções e bibliotecas são reutilizadas, por isso, há uma parte significativa dos códigos que foram feitos como bibliotecas e são usadas em todos os módulos.


\begin{figure}[h]
\centering
    \caption{Sistema Eletronônico - Robô Hospitalar (V2)}
    \centering % para centralizarmos a figura
    \includegraphics[width=17cm]{sistema_eletronico.png}
    \caption*{Fonte: Elaborada pelo autor}
    \label{figura:1° Versão Robô Hospitalar}
\end{figure}

\section{Software}

A área de computação da equipe, foi sem dúvidas a que mais se desenvolveu nos últimos seis meses. Na primeira versão do robô Hospitalar somente uma pessoa trabalhava, em contrapartida, na segunda versão três pessoas trabalham no software do robô, sendo duas delas inclusivamente para isso. De maneira geral, a confecção do software do robô hospitalar é dividida em cinco grandes frentes: Simulação, Algoritmos de Controle, Integração de Componentes, Visão Computacional e Desenvolvimento web e mobile.

A área de \textbf{Simulação}, feita com gazebo \cite{gazebo21} sem dúvida a melhor desenvolvida e consolidada nos últimos seis meses, que já foi quase que concluída, foi subdividida em três grandes problemas: modelo simplificado do robô, universo de simulação e Implementação de sensores. Esses três problemas já foram devidamente desenvolvidos e hoje em dia sofrem poucas alterações.

A área de \textbf{Desenvolvimento de algoritmos de controle} pode ser subdivida em desenvolver dois grandes problemas: Odometria e Navegação. A odometria consiste em pegar os dados corretamente dos encóderes do robô e navegação em conseguir fazer o robô ir de um ponto a outro de forma autônoma. Ainda está sendo desenvolvida essa parte da equipe, porque é sem dúvida o core de todo o robô.

A área de \textbf{Integração de Componentes} consiste em utilizar a Jetson Nano \cite{jetson21} com os algoritmos desenvolvidos e estabelecer uma conexão com os sistemas embarcados. 

A área de \textbf{Visão computacional} da equipe visa fazer o robô entender alguns objetos de ambientes hospitalares a partir de uma câmera. Por conta de muito trabalho nas outras áreas e a pouca aplicabilidade dela para o projeto hoje, essa parte de desenvolvimento não está com pessoas alocadas oficialmente.

A área de \textbf{Desenvolvimento Web e Mobile}, uma área nova na equipe, que foi adicionada com a entrada de uma nova integrante para o grupo. Essa área está em um momento de pesquisas constantes, porém é bem promissora para um futuro breve.  


\begin{figure}[h]
\centering
    \caption{Sistema Computacional - Robô Hospitalar (V2)}
    \centering % para centralizarmos a figura
    \includegraphics[width=17cm]{sistema_computacional.png}
    \caption*{Fonte: Elaborada pelo autor}
    \label{figura:1° Versão Robô Hospitalar}
\end{figure}


\end{document}